

%opening
\title{Sommersemester 2012 Nachklausur}
\maketitle

\section*{Aufgabe 1}
\subsection*{1.a}
    \begin{itemize}
        \item Programmbreite: Die Anzahl verschiedener Produktarten (zB verschiedene Pastasorten wie Spaghetti und Penne)
        \item Programmtiefe: Die Anzahl verschiedener Ausführungen der Produkte (zB Dinkel-, Weizen-, Spinatnudeln)
        \item Programmkonsistenz: Die Ähnlichkeit im Auftritt von Produkten (Verpackung, Wert, Nutzen, etc.)
    \end{itemize}

\subsection*{1.b}
    Dachmarke: Eine Marke für alle Produkte ermöglicht ein konsistentes Auftreten unter gleicher Verpackung, welches den Bekanntheitsgrad erhöht. \\
    Allerdings werden auch Imageschäden wie schlechter Geschmack sich auf die gesamte Produktpalette auswirken.

\subsection*{1.c}
\subsection*{1.d}
\subsection*{1.e}
\subsection*{1.f}
    \begin{itemize}
        \item Intensive Distribution - Das Produkt wird überall verkauft
        \item Selektive Distribution - Das Produkt wird nur bei Partnern verkauft, die den Anforderungen des Unternehmens genügen
        \item Exklusive Distribution - Ein Partner kriegt ein (regional) exklsuives Distributionsrecht
    \end{itemize}

\subsection*{1.g}
    Für unsere Nudeln macht eine selektive Distribution an Feinkostläden Sinn. So wird die Zugehörigkeit zum Spitzensegment noch einmal betont.

\subsection*{1.h}
    Vorteile:
    \begin{itemize}
        \item Absetzen von der Konkurrenz
        \item Erhöhen der Verkäufe
        \item Kundenbindung
    \end{itemize}

    Nachteile:
    \begin{itemize}
        \item Kundenbindung ggf. nur temporär
        \item Anbieter wird ggf. gezwungen die Preise dauerhaft zu senken
        \item Konkurrenz kann auch Preisaktionen machen (Vergeltung)
    \end{itemize}

\subsection*{1.i}
    Kundensicht:
    \begin{itemize}
        \item Minderung von Informationskosten - Bei Ausruf einer Sonderpreisaktion pos. beeinflusst
        \item Emotionaler Nutzen - nicht beeinflusst
    \end{itemize}

    Hersteller:
    \begin{itemize}
        \item Differenzierung - ggf. durch niedrigeren Preis ein Alleinstellungsmerkmal (positiv beeinflusst)
        \item Verhandlunsposition - positiv beeinflusst
    \end{itemize}

    Händler:
    \begin{itemize}
        \item Imagetransfer - Nicht beeinflusst
        \item Absatzrisiko - sinkt, also positiv beeinflusst
    \end{itemize}

\subsection*{1.j}
    \begin{itemize}
        \item Belohnung - Rabatte, Boni; Loyalität
        \item Zwang - Multichannel; Auslisten
        \item Legitimität - Verträge, Wettbewerbsrecht (beide Seiten)
        \item Expertise - After-Sales-Service; Verkaufspersonal
        \item Referenzwert - Marken, Premiumstrategie; Handelsmarken
    \end{itemize}



\section*{Aufgabe 2}
\subsection*{2.a}
    Vorteile:
    \begin{itemize}
        \item schnell
        \item günstig
        \item direkt
    \end{itemize}

    Nachteile:
    \begin{itemize}
        \item Unterschied zwischen tatsächlichem und angegebenem Verhalten
        \item Geschlossene Frageformen ungenau
    \end{itemize}

\subsection*{2.b}
    Nutzenorientierte Preisbestimmung kann in den Folien aus dem SS2020 nicht gefunden werden. \\
    Es gibt: Preisorientiert, Kundenorientiert, Wettbewerbsorientiert

\subsection*{2.c}
    AIDA-Modell kann nicht gefunden werden.

\subsection*{2.d und 2.e}
    Werbeelastizität kann nicht gefunden werden.

\subsection*{2.f}
    Die 4 Kategorien sind:
    \begin{itemize}
        \item Kontakte
        \item Aktivierung
        \item Geschwindigkeit
        \item Wirtschaftlichkeit
    \end{itemize}
    \ \\
    TV-Werbung bietet sich an. Die Zielgruppe besitzt in den allermeisten Fällen einen Fernseher und es wird eine große Menge an Kd. erreicht. \\
    Die Werbung aktiviert die audiovisuellen Sinne, der TV-Spot muss auch nicht sehr kompliziert sein (was ihn vergleichsweise günstig macht). \\
    Allerdings muss auch ein einfacher Spot erst produziert werden, was etwas Zeit kostet. Dafür kann man ihn lange nutzen und direkt nach der Produktion ausstrahlen.
    Die Transparenz ist nicht unbedigt gegeben und das Produkt kann auch nicht direkt ausprobiert werden (Handlungsbezug). Dies ist bei Rasenmähern aber eh nur sehr schwer möglich. \\
    \ \\
    