

%opening
\title{Sommersemester 2020 Hauptklausur}
\maketitle


\section{Aufgabe 1}
\subsection{1.a}
    Dyson ist eine DAchmarke, da alle Produkte unter dem gleichen Namen angeboten werden.

\subsection{1.b}
    Die segmentspezifische Ansprache ist die ausrichtugn einer Marke auf die Zielgruppe ihres Produktes/ihrer Produkte. \\
    Die Ausstrahlung einer Marke ist dei Auswirkung von Goodwill-/Badwill-Effekten auf andere Marken/Produkte des Unternehmens. \\
    \ \\
    Mit Dyson als Dachmarke wird keine starke segemtnspeiotifische Ansprache erreicht. Allerdings werden sich goodwill-Effekte direkt auf alle Produkte auswirken.

\subsection{1.c}
    Bei der Einführung von LED-Leuchten in das Dyson-Produktsortiment handelt es sich um eine Ljnineausweitung, da die bestehenden Linien (Staubsauger) weiter geführt werden und dei neuen Prtodukte unter der selben amrke angebotenw erden.

\subsection{1.d}
    Ein zentraler Faktor bei der Line Extension ist die Bekanntheit der Basisamrke und die damit verbundenen symbolischen und emotionalen Nutzen. \\
    Dyson ist im Staubsauergeschäft eine sehr bekannte Luxusmarke und durch die besondere Bauweise und gute funktionalität der Staubsauger auch mit einem hohen Nutzen verbunden.

\section{Aufgabe 2}
\subsection{2.a}
    Preissuche ist der Aufwand, sich über den Preis zu informieren. Dazu gehört nicht nur das lesen des Preises, sondern auch der Vergleich mit andern Produkten, erstellen eines Refernzpreises etc. (Preisinformationsaufnahme)\\
    \ \\
    Preiswissen ist das bei der Preissuche gespeicherte Wissen über den Preis. \\

\subsection{2.b}
    Dadurch das Kunden mehr auf den Preis achten werden die Aufwände in der Preissuche steigen. \\
    Damit man nicht immer neu auch den Preis achten mzuss wird auch die gespeicherte Menge an Wissen über den Preis steigen, also das Preiswissen.

\subsection{2.c}
    Dadurch dass die Preise von LIDL schon vor den Preisen anderer senkt, zeigt das Unternehemn dass es immer günstige Preise hat. Dadurch bleibt das Image des Preises gleich.
    
\subsection{2.d}
    da der Kunde nicht alle Preise kennt (oder kennen kann) wird das Preisimage durch einige wenige Preis beeinflusst. Dabei werden häufig Eckartikel wie Schokolade genommen, um die Preis ezu vergleichn.

\subsection{2.e}
    EDLP-Strategie beduetet dass kaum Rabatte gewährt werden, da die Preise jeden Tag so gering wie möglich sind. \\
    Bei Preis-Promos werden verschiedene Preise abwechselnd gesenkt, um Kundenverhalten zu steuern.

\subsection{2.f}
    Eine langfristige chance ist es, Kundenkaufverhalten zu steuern und damit das Warensortiemtn besser an das Kaufverhalten anzupassen. \\
    Auf der anderen Siete kaufen Kunden nur ein wenn sie eine Promotion kaufen. Lässt man das Konzept Promotions irgendwann weg (zB weil es sich nciht mehr lohnt), so werden die Kunden auch nciths mehgr kaufen.

\section{Aufgabe 3}
\subsection{3.a}
    Die Google-Anzeigen von eBay nutzen die periphäre Route, da das Involvement der Kunden, mehr über eine Handelksplattform zu lerenen, sehr gring ist. \\
    Durch den bloßen Kontakt durch das häufige Aufrufen der eBay Werbung durch Google wird die einstellung der Kunden verändert.

\subsection{3.b}
    Als Onlineunternehmen sollte eBay auch auf Webwerrbung setzen. Durch anmietung großer werbeflächen im internet kann das neue abrechnungssystem gut erklärt werden und durch einen Knopf ("hier klicken um jehr zu erfahren") kann dem Kunden das neue System direkt gezeigt werden.

\subsection{3.c}
    Die Klicks auf Werbeanzeigen wurden reduziert, gleichzeitig steigen die Klicks auf die regulären Suchergebnisse. Die Gesamtmenge an klicks ist dabei relativ konstant geblieben.

\subsection{3.d}
    Beim Last-Click-Modell wird eine Konversion dem letzten Brührungspunkt eines Click-Weges zugeordnet.
    %THEMA Last click

\subsection{3.e}
    DEr Rausnahmeeffekt der google Anzeigen entspricht etwa 1/4 im vergleich zu den 3/4 der ergebnisse. \\
    Das Ergbnismuster würd enicht eingfangen werden, da die Rausnahme von Werbung die Attribution von ergbnissen ändern wird.

\subsection{3.f}
    Nein, da die Anzeigen nur Kunden von den Suchergebnissen wegtreiben, aber keine neuen Kunden fangen.

\section{Aufgabe 4}
\subsection{4.a}
    Im Offlinehandel kann ein Produkt vor dem Kauf ausporibert werden. Beim Onlinehandel muss das Produkt erst gekauft und dann ggf. retourniert werden.
    
\subsection{4.b}
    \begin{enumerate}
        \item Kannibalisierung: Ein Verkaufskanal kann die Kunden von einem anderen Verkaufskanal wegziehen. Hier zB Online von Offline.
        \item Verwirrung: Unterschiedliche Return Policies können den Kunden verwirren, wo es für ihn besser ist einzukaufen.
    \end{enumerate}

\subsection{4.c}
    Die rel. Transaktionskosten im Onlinehandel/Versandhandel werden durch kürzere Lieferzeiten verringert.
    %TODO Konkreter

\subsection{4.d}
    Bei verkürzten Lieferzeiten kaufen mehr Kunden ein, die Marke wird bekannter. Dadurch sind auch mehr Menschen geneigt, die Läden zu nutzen.

\subsection{4.e}
    Bestandskunden sind an die Lieferzeiten gewöhnt. Sie können höchstens durch die nun kürzeren Lieferzeiten nochmals dazu verleitet werden, einzukaufen. \\
    Neue Kunden können auf der anderen Seite durch gute Werbemaßnahmen dazu verleitet werden, einen Service den sie vorher aufgrund lamger Lieferzeiten nicht in Betracht gezogen haben auszuprobieren. \\