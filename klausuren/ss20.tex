\documentclass{article}
\usepackage[ngerman]{babel}
\usepackage[utf8]{inputenc}

\title{Sommersemester 2020 Hauptklausur}
\begin{document}
\maketitle


\section{Aufgabe 1}
\subsection{1.a}
Es handelt sich bei Dyson um Einzelmarkenstrategie. Dyson bietet einen Produkttyp (Staubsauger) an, ist aber einer der Marktführer in der eigenen Kategorie (Beutellose Staubsauger).


\subsection{1.b}
\subsubsection{segmentspezifische Ansprache}
Bezeichnet die Eigenschaft des Unternehmens, nur in einem bestimmten Segment an Waren, Vertreter zu sein. Bei Dyson könnte dieses Segment zB Haushaltsgeräte - Staubsauger sein. Für die Einzelmarkenstrategie von Dyson ist die segmentspezifische Ansprache sehr gut umsetzbar, da sie ja ausschließlich für das eine Segment produzieren.

\subsubsection{Ausstrahlungseffekte}
Sind zB Goodwill- und Treuetransfers, wie unter anderem das Payback Punkte-System. Die Nutzung solcher Effekte ist für die Markenstrategie von Dyson nicht von Vorteil, da ja nur innerhalb der eigenen Produktkategorie Treuetransfers stattfinden könnten, die wenigsten Kunden aber mehr als einen Staubsauger brauchen.


\subsection{1.c}
Bei der Einführung der Dyson-Leuchten hadelt es sich um eine Linienerweiterung. Da die Leuchten noch unter der Marke Dyson zu haben sind, jedoch nicht in der Produktkategorie wie Staubsauger zu finden wären, ist es eine neue Produktlinie in gleicher Marke.


\subsection{1.d}
Ein zentraler Erfolgsfaktor ist, eine Basismarke mit hoher Bekanntheit zu haben. Dies ist bei Dyson der Fall ist, da sie durch ihren langjährigen Vertrieb von Staubsaugern sehr bekannt sind. Da Leichten im weiteren Sinne wie Staubsauger auch Haushaltswaren sind, werden Kunden diese Marke beim Einkauf wiedererkennen.


\section{Aufgabe 2}
\subsection{2.a}
\subsubsection{Preissuche}
Bezeichnet die Bemühung eines Kunden, sich plattformunabhängig Preisinformationen über verschiedene Produkte zu suchen. Das Suchen online von Alternativen zu einem gewissen Produkt und Preisvergleiche zweier gleicher Produkte verschiedener Hersteller fallen unter anderem darunter.

\subsubsection{Preiswissen}
Umfasst sämtliche preisbezogene Infos, die im Langzeitgedächtnis eines Nachfragers gespeichert sind. Dies wird aufgeteilt in explizites Preiswissen, also die Preisinformation, an die sich der Nachfrager bewusst erinnert und in implizites Preiswissen, welches die unbewusste Anwendung gespeicherter Informationen zu einem Produkt bezeichnet.

\subsection{2.b}
Da die Kunden preissensibler agieren, suchen sie auch eher nach verschiedenen Preisen. Man kann also demnach einen generellen Anstieg bei potentiellen Kunden in der Preissuche voraussetzen und damit einhergehend auch, dass Nachfrager ein größeres Preiswissen haben, da sie sich vor dem Kauf/Nachfragen eingehend über das Produkt und Alternativen informiert haben.

\subsection{2.c}
Das Preisimage umfasst die käuferindividuelle Bewertung des Preisniveaus der Einkaufsstätte. Sekt Lidl die Preise schon vor der Mehrwertsteuersenkung, so kommt bei Kunden an, dass Lidl unabhängig von den staatlichen Beschlüssen, den Kunden Rabatte gönnen möchte. Somit steigt das Preisimage bei den Kunden, da Lidl als günstiger eingestuft wird, sich das Produktsortiment aber nicht verschlechtert hat.

\subsection{2.d}
Da Lidl zB ein großes Warensortiment hat, kann kein Kunde alle Preise kennen. Durch einzelne Eindrücke aus dem Sortiment bildet sich ein Kunde einen Eindruck vom Preisimage. Dabei spielen vor allem die Preisgünstigkeit und die Preiswürdigkeit eine große Rolle. Je größer das angebotene Sortiment ist, desto höher ist das Preisimage.
Lidl kann dies zB nutzen, indem sie häufig gekaufte Artikel reduzieren, sodass viele Kunden diese Rabattaktion mitbekommen. Somit stiege das Preisimage, da die ausgewählten und von vielen Kunden gekauften Artikel das Preisimage verändern.

\subsection{2.e}
Die Stategie "`Every-Day-Low-Price"', welche viele Discounter nutzen, baut darauf auf, alle normalen Preise des Warensortiments niedrig zu halten. Vor allem häufig Verkaufte Produkte werden standardmäßig zu "`Tiefpreisen"' (solchen, die unter den aktuellen Marktpreisen liegen) angeboten, sodass die Konkurrenz von Vollsortimentern durch das Preisangebot ausgeschlagen wird.
Preis-Promotions sind hingegen nur zeitlich begrenzte Angebote, zB Rabattaktionen auf einzelne Produkte, die den Absatz bei Händlern kurzfristig fördern. Diese können auch auf den Standardpreis von Discountwaren gezählt werden.

\subsection{2.f}
\subsubsection{Chance}
Eine Chance von Preis-Promotions ist die kurzfristige Erhöhung des Absatzes bei Edeka, da durch das Sonderangebot mehr Kunden einen Kaufzwang für das reduzierte Produkt empfinden und es auch kaufen, wenn sie es eigentlich nicht benötigen würden.

\subsubsection{Risiko}
Bei andauernden Preis-Promotions kann die Glaubwürdigkeit von Edeka sinken. Dann würden Kunden die Qualität der angebotenen Produkte infrage stellen, da diese ja die ganze Zeit zu "`Schleuderpreisen"' angeboten werden und lieber zur Konkurrenz gehen.

\section{Aufgabe 3}
\subsection{3.a} %LS: Hier bin ich mir nicht ganz sicher/ etwas halbherzige Begründung
eBays Suchmaschinenanzeigen greifen die Beobachter über die periphere Route an. Die Anzeigen motivieren browsende Personen, auf die Website zu gehen und bieten ihnen eine spontane Gelegenheit, sich abzulenken. Kunden nehmen die Information wahrscheinlich nur am Rande ihrer Aufmerksamkeit wahr, da sie damit beschäftigt sind, die Informationen der Suchmaschine zu verarbeiten. 

\subsection{3.b}
Da eBay ein Online Anbieter ist, sollte eBay Online-Werbung nutzen, um Neuerungen zu kommunizieren. Dadurch bietet sich auch der Vorteil, dass audiovisuell sehr präzise und zügig erklärt werden kann, was sich ändert. Außerdem können sowohl auf der eigenen Plattform, als auch auf hausfremden Websites Werbung geschaltet werden.

\subsection{3.c}
Die bezahlten und die regulären Klicks gehen im Zeitraum Anfang Juli 2012 bis Anfang August 2012 auseinander. Die blauen Werbeklicks liegen also nicht nur generell deutlich unter den Regulären, sondern verlieren temporär auch an Bedeutung.
Oder: Es sieht aus wie ein Tal vs. es sieht aus wie ein Berg???

\subsection{3.d}
Grundlegend geht es bei der "`Last Click"'-Heuristik darum, dass nur der letzte Klick auf die Werbung eines Werbenden in einer Suchanfragenkette eines Suchenden dem Werbenden auch monetär gutgeschrieben wird.
Hier würden die Suchwortanzeigen überschätzt werden, da diese offensichtlich nicht so ausschlaggebend für den Aufruf von eBay sind, wie die unbezahlten Verlinkungen.

\subsection{3.e}
Wenn die Google-Suchwortanzeigen komplett rausgenommen werden würden, würde sich das Muster wahrscheinlich nicht viel ändern. Da die unbezahlten Verlinkungen zu eBay ohnehin Überwiegen, würde die Plattform auch ohne bezahlte Suchwortanzeigen noch viele Klicks einfangen.

\subsection{3.f}
Aufgrund der eben genannten Faktoren, ist es für eBay nicht nötig, Suchwortanzeigen zu schalten. Das Unternehmen bekommt durch die unbezahlten Verlinkungen sogar mehr Klicks und kann sich durch das Auslassen von Suchwortanzeigen Geld sparen.

\end{document}