

%opening
\title{Sommersemester 2013 Hauptklausur}
\maketitle

\section*{Aufgabe 1}
\subsection*{1.b}
    Durch den durchweg hohen Teilnutzenwert ist die Grillfläche das wichtigste Merkmal. \\
    Die Form wird generell gering bewertet, ist also das unwichtigste Merkmal. \\
    \ \\
    Der BBQ-QQ1 schneidet bei der Grillfläche nicht perfekt ab, eine etwas größere Grillfäche würde den Nutzen stark erhöhen. \\
    Bei der Form ist es eigentlich egal, welche man wählt. Die Werte sind sehr nah beieinander.

\subsection*{1.c}
    Keine Ahnung wie man so was berechnet aber ich habs mal so gemacht: \\
    \ \\
    Die Summe der Teilnutzenwerte ist $14+12+14+4+16 = 60$ bei einem Preis von $150€$. (1 Pkt = 2,50€) \\
    Die Summe ohne Marke ist $8+9+7+3+0 = 27$ bei einem Preis von $50€$. (1 Pkt. = 1,85€) \\
    \ \\
    Die Markenunterschied ist pro Punkt also $2,50 - 1,85 = 0.65€$. Ein No-Name-Grill mit 60 Pkte. würde dabei $1,85€ \times 60 = 111€$ kosten. \\
    Die Marke bringt also 39€ für diesen Grill ein.

\subsection*{1.d}
    Der monetäre Markenwert ist wichtig um Preise mit Bezug auf No-Name-Produkte festzulegen.
    Das Markenprodukt darf nur den Markenwert mehr kosten als ein vergleichbares No-Name-Produkt. \\
    \ \\
    Ausserdem ist der monetäre Markenwert wichtig in der bestimmung des Marktwertes des Unternehmens.

\subsection*{1.e}
    Direkte Kundenbefragung und Van Westendorp-Methode

\section*{Aufgabe 2}
\subsection*{2.a}
    Das ELM zeigt die verschiedenen Wege der Informationsverarbeitung bei Werbung. Es gibt die zentrale und periphäre Route. \\
    Die zentrale Route wird bspw. für eine hohe Motivation, sich mit dem Produkt auseinanderzusetzen, genommen.
    Ist die Einstellung der Konsumenten eher gering, so wird die periphäre Route genommen. \\
    \ \\
    Die Werbung muss auf die jeweilige Route angepasst werden. \\
    \ \\
    Agentur 2 wählt die zentrale Route auf Grundlage von Kognitionen. Dem Konsumenten wird durch den Chefkoch die Qualität und Glaubwürdigkeit der Argumente gezeigt.
    Die Werbung richtet sich auch hauptsächlich an Grill-Enthusiasten. \\
    \ \\
    Agentur 1 wählt die periphere Route auf Affektgrundlage. Der Konsument wird klassisch Konditioniert, den Grill mit Party und Gesellschaft zu verbinden. \\
    Die Motivation ist gering denn die Werbung zielt nicht auf Grillenthusiasten ab, sondern auf Amateur-Grillmeister.

\subsection*{2.b}
    "Informationsökonomie" nicht gefunden in SS2020 Folien

\section*{Aufgabe 3}
\subsection*{3.a}
    BBQ-QB1 sollte höherpreisig sein. Da die Konkurrezn sein Produkt für fast 50€ mehr anbietet, kann der Preis des QB1 leicht erhöht werden. Um Preisschwellen zu berücksichtigen würde ich einen Preis von zB 229,00€ wählen. \\
    \ \\
    BB3 ist deutlich höher als der Referenzpreis eines anderen Anbieters, allerdings liegt dieser nur knapp über den Herstellungskosten. Auf der anderen Seite ist 99,00€ eine sehr mächtige Preisschwelle, deswegen sollte der Grill nicht knapp über den 99€ platziert werden. \\
    Eine bepreisung mit 99,00 schmälert zwar Gewinne aber macht price matching mit der Konkurrenz. \\
    \ \\
    BQ9 kann analog zum QB1 seine Preise leicht erhöhen, so lange er noch günstiger als die Konkurrenz ist. Andernfalls kann auch hier wieder auf einen hohen Gewinn verziechtet und die 99,00€-Preisschwelle ausgenutzt werden. \\

\subsection*{3.b}
    Dies ist eine zetliche Preisdifferenzierung.

\subsection*{3.c}
    Ein Vorteil ist die Ausnutzung verschiedener Zahlungsbereitschaften der Kunden über das Jahr. \\
    Grills werden jedoch nicht oft gekauft und so kann es sein dass alle pot. Kunden einen Grill günstig zur Winterzeit kaufen und ihn bis zum Sommer nicht nutzen. \\

\subsection*{3.d}
    Ich konnte in den Folien keine Bewertungs-Kriterien finden...



\section*{Aufgabe 4}
\subsection*{4.a}
    Ein Verkauf über den einzelhandel machjt Sinn, es entstehen Handelskosten durch den notwendigen Gewinn des Händlers und dessen Produktions- sowie Transaktionskosten. Diese sidn aber geringer als die gesparten Transaktionskosten durch verzeicht auf Direktverkauf. \\
    \ \\
    Ein Vertrieb direkt über den einzelhandel macht Sinn, da Grillwaren hauptsächlich von Privatpersonen gekauft werden und somit die Handelskosten des Großhandels umgangen werden können. \\
    Ausserdem sind die TAK beim Verkauf durch einen Händler geringer als beim Direktverkauf.

\subsection*{4.b}
    ABC-Analyse kam nicht vor