

%opening
\title{Sommersemester 2014 Hauptklausur}
\maketitle


\section*{Aufgabe 1}
\subsection*{1.abcd}
    Nicht gefunden

\section*{Aufgabe 2}
\subsection*{2.a}
    Target Costing ist die Festlegung eines maximalen Verkaufspreises und die Anpassung der Produktions- und Marketingkosten an diesen Preis, sodass noch ein Gewinn erwirtschaftet wird. \\
    Auf der anderen Seite ist Kosten-Plus-Verfahren: Hierbei wird der Preis aus der Summe der Produktions- und Marketingkosten sowie dem Gewinn errechnet.

\subsection*{2.b}
    \begin{itemize}
        \item Billigprodukt: $5+3+0+0+0+22 = 30$
        \item A5000: $10+2+3+1+3 = 19$
        \item Luxusprodukt: $20+8+6+0+3+2 = 39$
    \end{itemize}

    Mit den Teilnutzenwerten 22 und 2 und den Preisen 200 und 600 würde ich einen Preis anstreben, der in der Mitte liegt: 12 Nutzenwertpkte. \\
    Das entspräche einem Preis von 400€.

\subsection*{2.c}
    Eine Conjoint Analysis als Preisgrundlage ist nicht sinnvoll. Bei der conjoint Analysis werden nur vordefinierte Merkmale betrachtet, ggf. sind das gar nicht die Merkmale die den Kunden wirklich wichtig sind. \\
    Ausserdem wird der Preis als Merkmal in den Hintergrund gerückt, was das Preisbewusstsein der Befragten schwächt.

\subsection*{2.d}
    Die direkte Kundenbefragung, dabei werden die Preise genau beachtet, was bei einem Haushalktsgerät wie einem Kühlschrank auch sehr typisch ist. \\
    Durch offene Frameformen kann ein guter Preis auch valide geschätzt werden. Das Problem des Variable Menge-Falls tritt bei Kühlschrankkäufen eigentlicha uch nicht ein.

\subsection*{2.e}
    Der durchweg höchste Teilnutzen ist der Inhalt (in L). bei 2 von 3 Produkten liegt der Wert für den Inhalt über allen anderen Werten

\subsection*{2.f}
    %THEMA habe ich nicht gefunden
    %Langfristig  möchte  die  Firma  Eiszeit  neue,  innovative  Produktideen  umsetzen,  um nicht  den  Anschluss  an  Konkurrenten  zu  verlieren. Welches  unternehmensexterne Verfahren  zur  Generierung  neuer  Produktideenist  hier  am  besten?Was  sind  jeweils zwei Vor-und Nachteile? (3 Punkte)


\section*{Aufgabe 3}
\subsection*{3.a}
    %TODO Skizze einfügen
    Das ELM hat 2 Wege, den primären und den periphären Weg: \\
    Welcher Weg genommen wird um die Einstellung von Kunden zu ändern hängt von der Motivation des Kudnen ab, sich mit dem Produkt zu beschäftigen. \\
    Bei high involvemtn wird der primäre weg genommen, bei low involvement der periphäre.

\subsection*{3.b}
    Motivation, Fähigkeit, Gelegenheit ... 
    \begin{itemize}
        \item mehr: zentrale Route
        \item weniger: Periphäre Route
    \end{itemize}

    Grundlage der Einstellung ... 
    \begin{itemize}
        \item Affekt
        \item Koginitonen
    \end{itemize}
    \ \\
    Die Nivea-Werbung ist eine positive Einstellung zur Kommunikationsmaßnahme, die Werbung soll emotional verarebitet werden. \\
    Die TecXL-Werbung arbeitet auf Grundlage von Kognitionen. Mit guten wertbasierten Argumenten, glaubhaft präsentiert.

\subsection*{3.c}
    Die Nivea-Werbung ja, die TecXL-Werbung ist zu klein geschrieben mit zu viel Informationen. Bei der Nivea-Werbung ist nur nicht ganz klar wofür geworben wird.


\section*{Aufgabe 4}
\subsection*{4.abc}
    %THEMA Informationsasymmetrie

\subsection*{4.de}
    %THEMA Mitarbeitermotivation
