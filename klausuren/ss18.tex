


%opening
\title{Sommersemester 2019 Nachklausur}
\maketitle



\section{Aufgabe 1}
\subsection{1.a}
    Airbnb ist allgemein im Niedrigpreissektor anzutreffe. Man kriegt fast keinen Service, dafür sind die Preise pro Übernachtung sehr günstig. \\
    Airbnb Plus würde ich im mittleren Preissegment sehen. Man kriegt mehr Leistung und Sicherheit als bei den niedrigen Preisen des "allgemeinen" Airbnb, es ist aber noch keine Luxus-Unterkunft. \\
    \ \\
    Ich denke die Position wird sich nicht halten lassen, da die Hotelbranche zu ähnlichen Preisen einen besseren und persönlicheren Service bietet als Airbnb Plus. \\

\subsection{1.b}
    Ein konsistenter Markenauftritt ist wichtig, da nur durch Übung, Erinnerung und Assoziation eine Information im Langzeitgedächtnis bleibt. \\
    Es muss also immer wieder durch einen konsistenten Auftritt eine Kodierung der Information im Hirn erreicht werden.

\subsection{1.c}
    Airbnb Plus soll ein anderes Segment an Kunden bedienen als Airbnb. Dazu wird die Preisdifferenzieung auf Basis der Leistung angewandt. Mit Airbnb Plus erhält man besseren Service als bei Airbnb normal. \\
    Ausserdem kann räumlich Differenziert werden: Wenn die besten Wohnung im Stadtzentzum alle mit Airbnb Plus angeboten werden, ist man dazu geneigt den Service zu nutzen um so eine Wohnung zu bekommen.
    %TODO stimmt das so?

\subsection{1.d}
    Die MArkenidentität ist die Assoziation der Marke mit menschlichen attributen (freundlich, nett, sportlich, etc.) \\
    \ \\
    Das Markenerlebnis ist der Auftritt der Marke nach Außen, zB durch Verpackung, angebotene Leistung, Farben, Logo. \\
    \ \\
    Die Identität von Airbnb ist nicht besonders zuverlässig. "billig" und "unsicher" sind in meinen Augen Hauptattribute der Marke Airbnb. \\
    Das passt nicht zum geplanten Erlebnis, bei dem man durch komptente Mitarbeiter betreut werden soll und vorher durchgecheckte Wohnung angeboten bekommt.

\subsection{1.e}
    Airbnb Plus zielt hauptsächlich auf die Risikoreduktion ab. Wie in Aufgabe 1.d) erläutert wird das Angebot von Airbnb als nicht besonders sicher oder zuverlässig angesehen. \\
    Durch die Betreuung durch Mitarbeiter und der vorherige Check-Up der Wohnung wird das risiko, ein schlechtes Produkt zu erhalten, reduziert.



\section{Aufgabe 2}
\subsection{2.a}
    Die Kosten-Plus-Preisbildung nimmt alle Kosten (Produktion, Rohstoffe, Transport, Werbung) und den geplanten Gewinn, addiert diese auf und teilt sie durch die erwartete Absatzmenge. \\
    So wird der Preis pro Einheit errechnet. \\
    Ein Vorteil ist die sehr Transparente Preiserstellung, der Preis wird also sehr fair angesehen. Auf der anderen Seite wird so das "over-engineering" nicht berücksichtigt, welches sich negativ auf den Preis auswirken kann.

\subsection{2.b}
    Durch eine direkte Kundenbefragung mit offenen und geschlossenen Frageformen kann die Zahlungsbereitschaft abgefragt werden. Dabei werden den Kunden Fragen zum Produkt und zum Preis gestellt,
    bspw. "Würden sie das Produkt zu einem Preis von xx€ kaufen?" (geschl.) oder "Was ist der maximale Preis in € bei dem sie das Produkt noch kaufen würden?" (offen)

\subsection{2.c}
    Referenzpreise sind die Preise vergleichbarer Konkurrenzprodukte, an denen man das Preisniveau ablesen kann. \\
    Als Referenzpreis kann das Trikot eines anderen Fussballvereins gesehen werden. Problem bei Referenzen von Triokots sind, dass die jeweiligen Trikots hauptsächlich in einem Land verkauft werden
    (zB brasilianische Triokots in Brasilien) und sich die ZBs der einzlnen Länder stark unterscheiden.

\subsection{2.d}
    Ja, da das Trikot sich nächstes Jahr wieder ändern wird kann der zentrale Vorteil von Preispromos genutzt werden, nämlich der erhöhte Absatz der trikots zu günstigen Preisen. \\
    Der Nachteil, dass das Preisniveau sinken wird und Kunden die Waren nur im Angebot kaufen besteht eher nicht, da nächste WM jeder wieder ein Trikot zu Beginn des Cups braucht.

\subsection{2.e}
    Die Preisfairness ist die Einschätzung des Preises durch den Kunden. \\
    Der Kunde erẃartet dass der Anbieter ihm ein gutes Angebotr macht, gesteht dem Anbieter aber auch die Absicht ein, 
    Gewinn zu machen. So muss ein Preis gefunden werden, welcher für beide Seiten nachvollziehbar ist.

\subsection{2.f}
    Wie in der e) bereits genannt, sieht jede Seite es als ihr Recht an, einen guten Deal zu finden. \\
    Preiserhöhungen ohne erhöhte Unkosten durch den Anbieter werden allgemein als unfair angesehen. 
    Da Adidas aber betont, die Kosten der Produktiuon eines Trikots haben sich erhöht, 
    kann der Preis gerechtfertigt werden und wird damit wieder als fair angesehen.



\section{Aufgabe 3}
\subsection{3.a}
    Influencer-Marketing nutzt sog. Influencer, Social Media Nutzer mit großem Impact, um die Kaufentscheidungen von Kunde zu steuern. \\
    Dabei ist der Vorteil, dass die Influencer aus Kundensicht als Person aus "unserer Mitte" gesehn werden. Dadurch steigt ihre Glaubwüridgkeit.
    So ist das auch bei Bibi und MK.

\subsection{3.b}
    Durch den Klick auf einen Affiliatelink kann sehr gut gemessen werden, woher die Kunden kommen. \\
    Es ist in etwa so wie ein Aktionscode, nur hat der Kunde nichts davon.

\subsection{3.c}
    TV-Werbung, denn durch die Nutzung eines guten TV-Spots kann auch eine emotionbale Assoziation erreicht werden. \\
    Aussenwerbung, denn auch diese setzt auf die periphere Route durch bloßen Kontakt mit der Marke. \\
    
\subsection{3.d}
    Nein, da Bibi nicht gut zur hochpreisigen Marke MK passt. Bibi passt gut zu ihrer Zielgruppe, da dies aber meiste Kinde rund Jegendliche ohne Geld sind passen diese nicht zu MK. \\
    Die Zielgruppe von MK sind junge Erwachsene mit zu viel Geld, diese werden aber nicht die Videos von Bibi angucken.

\subsection{3.e}
    Der Trouble-Faktor ist das risiko bei der Nutzung eines Testimonials. 
    Es beschreibt die Möglichkeit dass das Tesitmonial etwas macht, was die öffentliche Meinung von ihm verschlechtert.
    Ein Beispiel anhand von Bibi wäre zum Beispiel wenn sie ihren Partner betrügt und das öffentlich wird. Das Image von Bibi nimmt Schaden und damit auch das Image von MK.

\subsection{3.f}
    Die Chance der Wiedererkennung der Marke steigt, wenn die Marke eine starke Präsenz hat. Bei einer Marke, welche Lifestyleprodukte anbietet, ist dies sehr wichtig. Product Placement ist dabei eine sehr kostengünstige Methode, eine hohe Bekanntheit der Marke zu erreichen.\\
    Gleichzeitig passt das Product Placement nicht unbedigt zum Image der Exklusivität der Luxusmarke MK. Wenn jeder mit den Produkten rumrennt spricht das nicht für eine besondere Exklusivität. \\
    \ \\
    Wenn das Placement gut geregelt ist und nicht zu aufdringlich kann es gemacht werden. Das Risiko des Verluistes der Exklusivität überwiegt in meinen Augen allerdings den Vorteil günstiger Werbung durch PP.