

%opening
\title{Sommersemester 2017 Nachklausur}
\maketitle



\section{Aufgabe 1}
\subsection{1.abc}
    %THEMA LEbens zyklus

\subsection{1.d}
    Der Markenkern ist die Seele der Marke, der Gedanke der das zentrale Mantra einfängt.

\subsection{1.e}
    Die Elektromobilitätsstrategie muss dem Markenkern folgen, also muss BMW probieren den Fahrspaß bei Elektroautos beizubehalten.

\subsection{1.f}
    Die MArkenarchitektur ist die Auordnung von Produkten eines Unternehmens zu den Marken des Unternehmens sowie die Bezeihung der Marken unterinander.

\subsection{1.g}
    \begin{itemize}
        \item Line Extension ist die Ausweitung der Marke auf neue Produkte oder Geschäftsfelder. Ein Beispiel wäre es die bestehenden Modelle von BMW (3er 5er etc.) auch als Elektroauto anzubieten.
        \item Brand Extension ist der Transfer der Marke auf ein neues Geschäftsfeld. Beispiel wäre wenn BMW nur noch Elektroautos anbietet.
        \item Multibranding ist das einführen einer neuen/fremden Marke für gemeinsame Produkte. Ein Beispiel wäre "BMW \& Bosch" für Elektroasutos beider Hersteller
        \item Entwicklung neuer Marken und Linien baut eine ganz neue Marke für das neue Produkt auf. Beispiel wäre BEMW - Bayrische Elektromotorenwerke für Elektroautos
    \end{itemize}

\subsection{1.h}
    Die BMW i-Modelle werden immer noch unter der Marke BMW neben den bestehenden Modellen angeboten, gehört als zur Line extension.

\subsection{1.i}
    %THEMA Markenreichweite



\section{Aufgabe 2}
\subsection{2.a}
    Es gibt die kostenorientierte, kundenorientierte und wettberwerbsorientierte Preisbildung. \\
    Die Preisbildung von Uber ist kundenorientiert, da der gezahlte Preis nur von der Zahlungsbereitschaft der Kunden abhängt.

\subsection{2.b}
    Preisfairness ist die Kunden- sowie Anbieterseitige Bewertung des Preises als gerechtfertigt. \\
    \ \\
    Die oben genannte Transparenz ist ein Einflussfaktor auf die fairness. Transparente Preise werden generell als fairer angesehen. \\
    Auf der anderen Seite ist die Abschöpfung der Zahlungsbereitschaft ein Faktor: Kunden die mehr zahlen können werden höher bepreist bzw Kunden die weniger zahlen können müssen nicht so viel zahlen. \\

\subsection{2.c}
    Nein, da der Preis von vielen Faktoren abhängig ist (Route, Zeit, Ort, etc.) kann die Qualität zweier Fahrten nicht sehr gut nur anhand des Preises vergleichen werden. \\
    Bei Uber wird sehr viel Wert auf Kundenbewertungen der Fahrt gelegt, so kann man nach jeder Fahrt anhand mehrerer Kriterien den Fahrer bewerten. \\

\subsection{2.d}
    Die vollständige Abschöpfung der Zahlungsbereitschaft aller Kunden. \\
    Manche Kunden würden das Produkt nicht kaufen wenn der Preis höher wäre, diesen Kunden wird dann ein niedrigerer Preis angeboten. \\
    Andersherum wird Kunden, die eine höhere Zahlungsbereitschaft haben als der Ausgangspreis, ein etwas teurerer Preis angeboten.

\subsection{2.e}
    Es handelt sich um den 1. Grad der Differenzierung: Die Kudnen werden individuell unterteilt und die persönliche maximale ZB abgeschöpft.

\subsection{2.f}
    \begin{itemize}
        \item Nichtübertragbarkeit: Gegeben, da eine Fahrt nicht übergeben werden kann. So kann der Start- und Zielpunkt bspw. nicht geändert werden
        \item Wahrnehmung als fair: Hängt von der Reaktion der Kunden ab, aber wenn man nur Aufgrund seiner Position gezungen wird mehr zu zahlen ist das nicht besonders fair.
        \item Segmentierungsmöglichkeit: Gut gegeben, da jeder einen Account hat und die App genug Daten sammelt um Segmente bilden zu können
    \end{itemize}



\section{Aufgabe 3}
\subsection{3.a}
    Die Online-Branche setzt drauf, dass die angebotenen Produkte/Diensteleistungen direkt ausprobiert werden können. \\
    Da man meist zuhause TV guckt und dort auch seinen PC stehen hat (oder zumindest mit dem WiFi verbunden ist) macht es für Online-Dienstleister Sinn,
    TV-Werbung zu schalten.

\subsection{3.b}
    \begin{itemize}
        \item Kontakte: Die Anzahl Kontakte ist bei TV-Werbung generell hoch, bei Onlinewerbung hängt es von der Plattform ab. Mit personalsierter Werbung werden noch weniger Leute getroffen, allerdings sinkt der Streuverlust
        \item Aktivierung: Die Aktivierung ist bei beiden Werbeformen sehr schlecht. Weder bei TV-Werbung noch bei Onlinewerbung ist die Aufmerksamkeit besonders hoch.
        \item Geschwindigkeit: Onlinewerbung mit Bannern ähnelt Printwerbung, sie ist sehr schnell herstell- und verteilbar. Ein TV-Spot muss erst aufwendig bestellt und produziert werden.
        \item Wirtschaftlichkeit: Onlinewerbung mit Bannern ist sehr kostengünstig und durch den geringen Streueffekt (s.o.) sehr kosteneffektiv. Ein TV-Spot kostet sehr viel in der Produktion
    \end{itemize}
    
\subsection{3.c}
    Die Radiowerbung nutzt die periphere Route, da kein hohes Involvement vorausgesetzt wird. 
    Man wird nur mit der Werbung und dem Produkt wiederholt in Kontakt gebracht. \\
    Auf der anderen Seite ist die Aussenwerbung auf hohes Involvement getrimmt.
    Durch den Slogan wird eine emotionale Bindung zu Alexa gestärkt (a la "es geht nicht mehr ohne")
    und durch den QR-Code können Interessierte direkt mehr Informationen erhalten. \\
    \ \\

\subsection{3.d}
    Radiowerbung eignet sich gut, da Radio hauptsächlich im Auto gehört wird und die Aufmerksamkeit des Fahrers nicht gegeben ist. \\
    Durch die Hohe Reichweite (bei Nutzung mehrerer Stationen) wird eine große Anzahl Kunden mit starken Streuverlust erreicht,
    es eignet sich also perfekt für eine Werbung die auf bloßen Kontakt abzielt.

\subsection{3.e}
    \begin{itemize}
        \item Recognition kann man durch ein Experiment, bei dem die Leute die App auf einem Gerät suchen müssen, messen.
        \item Aided Recall kann man durch erneutes Vorspielen der Werbung erreichen.
        \item Unaided Recall kann man durch befragung erreichen ("Wo/wie kann man Alexa auch einsetzen?")
    \end{itemize}