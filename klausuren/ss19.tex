\documentclass{article}
\usepackage[ngerman]{babel}
\usepackage[utf8]{inputenc}

\title{Sommersemester 2019 Nachklausur}
\begin{document}
\maketitle


\section{Aufgabe 1}
\subsection{1.a}
Der Markenwert setzt sich aus dem Preispremium im Vgl. zu identischen Nicht-Markenprodukten + der zusätzlich verkauften Menge im Vgl. zu identischen Nicht-Markenprodukten zusammen.  

\subsection{1.b}
Das BrandZ-Rating orientiert sich nicht ausschließlich an den Umsätzen des Unternehmens, sondern berechnet die Brand Contribution mit ein. Da die Konsumentenmeinung ggü. Amazon im Jahr 2019 offensichtlich gestiegen ist, ist auch das BrandZ-Rating deutlich mehr gestiegen, als nur um die Gewinne, die Amazon in diesem Jahr eingefahren hat.
Zwei konkrete Maßnahmen Amazons, die in die Steigerung der Konsumentenmeinung und somit eine bessere Brand Contribution mit einspielen sind Amazons als hervorragend bewerteter Kundenservice und das ständig wachsende Angebot an Produkten und Dienstleisungen.

\subsection{1.c} kundenseitiges Markenwissen ermitteln
\subsubsection{1. Herausforderung}
Bei den Kundenbefragungen zur Brand Contribution werden sehr spezifische Dinge abgefragt, zu denen Kunden absolut unterschiedliche Meinungen haben können. Vor allem, da "`Verlangen"' und "`Bequemlichkeit"' sehr subjektive Werte sind.
\subsubsection{2. Herausforderung}
Amazon-Kunden haben ggf. kein konkretes Wissen über die Marke Amazon, sondern fühlen sich nur aufgrund emotionaler Werte, welche durch gutes Marketing von Seiten Amazons hervorgerufen werden kann, zur Marke hingezogen. Dies ergibt keine wertvolle Einschätzung der Qualität der Marke gegenüber.

\subsection{1.d}
\subsubsection{Markenidentität}
Die Markenidentität sind zB das stets gleich aussehende "`amazon"'-Logo mit dem Smiley-Pfeil, welches in allen Diensten zu erkennen ist, sowie auch die Typographie in den ebenfalls überall vorhandenen Slogans. Die Konsistenz innerhalb dieser Designentscheidungen generiert ein eindrückliches Bild, das bei Betrachtern direkt eine Assoziation mit der Marke "`Amazon"' hervorruft.

\subsubsection{Markenerlebnis}
Das Markenerlebnis spiegelt sich darüber wieder, dass Amazon in allen Diensten nicht nur ein einheitliches Auftreten hat, sondern sich auch inhaltlich gleich präsentiert. zB hat der "`Haupt-Dienst"' Amazon einen stärkeren Slogan als die anderen Serviced (video und music). Für die Streamingdienste ist allerdings auch inhaltliche Konsistenz vorhanden, da die Slogans, sowie auch die Logos, sich sehr ähneln. Somit wird den Kunden über alle Berührungspunkte das Bild "`Amazon"' vermittelt.

\subsection{1.e}
Dadurch, dass die Amazon-Services einen so hohen Wiedererkennungswert haben, bleibt die Erinnerung an die Marke im Gedächtnis eines Konsumenten länger hängen. Dieser muss sich nämlich nicht unterschiedliche Logos oder Slogans merken, sondern kann sich anhand der eingängigen amazon-Typographie und des Logos einfach dieses als Anhaltspunkt merken und wird so jeden Amazon-Dienst wiedererkennen.

\subsection{1.f}
"`prime video"' und "`amazon music"' sind beides Line Extensions, da sie den Namen der Hauptmarke enthalten und auch deren Image mit übernehmen, jedoch in einen anderen Produkt-Sektor gehören. Würden sie als komplett neue Marke verkauft werden, wäre dies eine Brand Extension.


\section{Aufgabe 2}
\subsection{2.a}
\subsubsection{Kosten-Plus-Preisbildung}
Diese Preisbildung geht von der Ausgangsfrage aus "`Was wird ein Produkt kosten"' und betreibt dahingehend auch die Entwicklung und Kostenermittlung. So wird die Produktentwicklung immer wieder angepasst, sollte der daraus resultierende Produktpreis zu hoch sein, bis das Produkt schließlich einen annehmbaren Preis hat. Erst dann wird auch Produziert und Verkauft.\\
Nachteil für die Migräne-App: die App ist eine Art Einzelstück-Produktion, da sie nach fertigstellung nicht in Masse produziert, sondern nur online gestellt werden muss. Darauf lässt sich dieses Prinzip schlecht übertragen, da die Stückkosten der App sich an keinerlei Mengenproduktion und damit verbundener Aufwendungen orientieren.

\subsubsection{Target Costing}
Beim Target Costing ist die Ausgangsfragestellung "`Was darf ein Produkt kosten?"' und nach Erstellung der Anforderungen an das Produkt, wird direkt ein Zielpreis festgelegt, der fix bleibt. Daraufhin wird der Entwicklungsprozess soweit angepasst, bis er dem Zielpreis genügt. Erst dann wird produziert und verkauft.\\
Dies ist hier schwer einsetzbar, da die Entwicklung der App nicht an Material, sondern an Zeit gebunden ist. Solange die App nicht den Anforderungen aus dem Pflichtenheft entspricht, müssen die Entwickler weiterbeschäftigt werden. Dadurch kann der Zielpreis durchaus schnell überschritten werden, ohne, dass der Prozess groß angepasst werden kann - sonst wäre sie nämlich unfertig.

\subsection{2.b}
Ja, jedoch werden sich die wenigsten Kunden mit dem Prozess zur Erstellung einer Software genügend auskennen, um einschätzen zu können, wie viel diese Wert sein sollte. Deswegen sollte die Befragung mit geschlossenen Fragen vollzogen werden. Damit würde man den Kunden wichtige Informationen wie zB einen annehmbaren Preisbereich entlocken.

\subsection{2.c}
Dies ist die Preisdifferenzierung  1. Grades.

\subsection{2.d}
\subsubsection{Monopolistischer Spielraum}
Die App ist zwar Marktführer in der Sparte, es gibt aber noch genügend Konkurrenz. Daher erfüllt die App dies.
\subsubsection{Nichtübertragbarkeit}
Die App ist durch die persönlichen Einstellungen eines jeden Kunden schwer übertragbar, vor allem, da Migräne bei jedem Patienten anders verläuft. Daher erfüllt die App dies.
\subsubsection{Möglichkeit zur Kundensegmentierung}
Die Kunden können zB nach Häufigkeit des Auftretens von Migräneanfällen segmentiert werden, jedoch gibt es andernfalls wenige Möglichkeiten, die Kunden wirklich definit zu segmentieren. Daher erfüllt die App dies nicht.
\subsubsection{Wahrnehmung als fair}
Kunden könnten es als unfair empfinden, für einzelne Dienste innerhalb der App zahlen zu müssen, nachdem sie die App schon heruntergeladen und die Diagnose bereits gestartet haben. Daher erfüllt die App dies nicht.
\subsubsection{Verhältnismäßigkeit}
Dass jede In-App-Behandlung etwas kostet, könnten Kunden als unverhältnismäßig empfinden, da mit der App ja ein zusammenhängender medizinischer Dienst promotet wird und nicht ganz viele unabhängige. Daher erfüllt die App diese nicht.

\subsection{2.e}
Die nutzenorientierte Preisbestimmung orientiert sich am Gesamtnutzen des Produktes, welche aufgeteilt in verschiedene Teilnutzenaspekte und anhand dieser bewertet wird. Zwei Voraussetzungen dafür sind:
\begin{itemize}
	\item Das Produk lässt sich in Teilnutzenaspekte aufgliedern. Ist in diesem Fall gegeben, da ja zwei verschiedene Services bewertet werden können.
	\item TODO
\end{itemize}


\section{Aufgabe 3}
\subsection{3.a} 
Die zentrale Route. Die Markenbotschafterin im Werbespot ist weit bekannt, sieht schön aus und wird sinnlich dargestellt.
Dies fördert bei Zuschauern eine hohe Motivation, sich mit ihr auf die gleiche Stufe stellen zu wollen und das Parfum zu erwerben. Dies entspricht genau der Zielsetzung der zentralen Route des ELM - Kunden direkt anzusprechen, sodass diese sich unmittelbar mit dem Produkt identifizieren.

\subsection{3.b}
\subsubsection{Chancen}
\begin{itemize}
	\item Hohes Verlangen der Kunden nach dem Produkt durch das (unterbewusste) Empfinden von Lust 
	\item Durch den Einsatz eines so positiv bekannten Testimonials wird das Ansehen des Produktes gesteigert und auch als positiv wahrgenommen
\end{itemize}

\subsubsection{Risiken}
\begin{itemize}
	\item Abschreckung der Kunden durch den Einsatz eines für einige noch "`tabuisierten"' Themas
	\item Kunden können die Werbung und somit das Produkt als billig wahrnehmen, da Sexualisierung oft auch mit Realtiy-TV, Sexarbeit und anderen weniger eleganten Themen assoziiert wird
\end{itemize}


\subsection{3.c}
\subsubsection{Erzeugung von Aufmerksamkeit}
TV-Werbung erhält eine hohe Aufmerksamkeit, da viele Leute noch fernsehen. Vor allem zu Stoßzeiten (zB zur "`Prime Time) wird die Werbung eine hohe Reichweite haben. Da Charlize Theron selbst auch aus dem TV- und Filme-Mediansektor bekannt ist, stehen die Chancen hoch, dass sie weitreichend erkannt wird und somit eine positive Auswirkung für das Produkt erzielt - so, wie in b) gesagt.
\subsubsection{Ansprache der Sinne}
TV-Werbung ist audiovisuell und spricht Zuschauer auf direktem Wege an. Ganz so, wie es die zentrale Route des ELM vorsieht. Von daher eignet sie sich gut für die Dior-Kampagne.
\subsubsection{Handlungsbezug}
Der Handlungsbezug ist bei TV-Werbung nicht sehr hoch, da die Beobachter das Produkt nicht direkt ausprobieren können.

\subsection{3.d}
Charlize Theron ist, wie in c) schon erwähnt weitbekannt. Durch ihr Engagement als UN-Friedensbotschafterin und des Orcar-Titels wird sie als einflussreich und selbstbewusst wahrgenommen. Dass sie kein Problem hat, sich freizügig zu zeigen, unterstützt dieses Bild nur noch.
Von daher eignet sich Charlize Theron gut als Markentestimonial für Dior Jadore.

\subsection{3.e}
\subsubsection{Recognition}
Beispielfrage: "`Erkennen Sie die Frau auf dem Bild?"' und dazu ein Standbild von Theron aus dem Spot zeigen.
\subsubsection{Unaided Recall}
Beispielfrage: "`Was trägt die Frau am Anfang des Spots?"'
\subsubsection{Aided Recall}
Beispielfrage: "`Charlize Theron spielt in einer Dior Werbung mit - welches Parfum bewirbt sie?"'

\end{document}