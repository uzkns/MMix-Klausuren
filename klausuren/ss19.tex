\documentclass{article}
\usepackage[ngerman]{babel}
\usepackage[utf8]{inputenc}

\title{Sommersemester 2019 Nachklausur}
\begin{document}
\maketitle


\section{Aufgabe 1}
\subsection{1.a}
Der Markenwert setzt sich aus dem Preispremium im Vgl. zu identischen Nicht-Markenprodukten + der zusätzlich verkauften Menge im Vgl. zu identischen Nicht-Markenprodukten zusammen.  

\subsection{1.b}
Das BrandZ-Rating orientiert sich nicht ausschließlich an den Umsätzen des Unternehmens, sondern berechnet die Brand Contribution mit ein. Da die Konsumentenmeinung ggü. Amazon im Jahr 2019 offensichtlich gestiegen ist, ist auch das BrandZ-Rating deutlich mehr gestiegen, als nur um die Gewinne, die Amazon in diesem Jahr eingefahren hat.
Zwei konkrete Maßnahmen Amazons, die in die Steigerung der Konsumentenmeinung und somit eine bessere Brand Contribution mit einspielen sind Amazons als hervorragend bewerteter Kundenservice und das ständig wachsende Angebot an Produkten und Dienstleisungen.

\subsection{1.c} kundenseitiges Markenwissen ermitteln
\subsubsection{1. Herausforderung}
Bei den Kundenbefragungen zur Brand Contribution werden sehr spezifische Dinge abgefragt, zu denen Kunden absolut unterschiedliche Meinungen haben können. Vor allem, da "`Verlangen"' und "`Bequemlichkeit"' sehr subjektive Werte sind.
\subsubsection{2. Herausforderung}
Amazon-Kunden haben ggf. kein konkretes Wissen über die Marke Amazon, sondern fühlen sich nur aufgrund emotionaler Werte, welche durch gutes Marketing von Seiten Amazons hervorgerufen werden kann, zur Marke hingezogen. Dies ergibt keine wertvolle Einschätzung der Qualität der Marke gegenüber.

\subsection{1.d}
\subsubsection{Markenidentität}
Die Markenidentität sind zB das stets gleich aussehende "`amazon"'-Logo mit dem Smiley-Pfeil, welches in allen Diensten zu erkennen ist, sowie auch die Typographie in den ebenfalls überall vorhandenen Slogans. Die Konsistenz innerhalb dieser Designentscheidungen generiert ein eindrückliches Bild, das bei Betrachtern direkt eine Assoziation mit der Marke "`Amazon"' hervorruft.

\subsubsection{Markenerlebnis}
Das Markenerlebnis spiegelt sich darüber wieder, dass Amazon in allen Diensten nicht nur ein einheitliches Auftreten hat, sondern sich auch inhaltlich gleich präsentiert. zB hat der "`Haupt-Dienst"' Amazon einen stärkeren Slogan als die anderen Serviced (video und music). Für die Streamingdienste ist allerdings auch inhaltliche Konsistenz vorhanden, da die Slogans, sowie auch die Logos, sich sehr ähneln. Somit wird den Kunden über alle Berührungspunkte das Bild "`Amazon"' vermittelt.

\subsection{1.e}
Dadurch, dass die Amazon-Services einen so hohen Wiedererkennungswert haben, bleibt die Erinnerung an die Marke im Gedächtnis eines Konsumenten länger hängen. Dieser muss sich nämlich nicht unterschiedliche Logos oder Slogans merken, sondern kann sich anhand der eingängigen amazon-Typographie und des Logos einfach dieses als Anhaltspunkt merken und wird so jeden Amazon-Dienst wiedererkennen.

\subsection{1.f}
"`prime video"' und "`amazon music"' sind beides Line Extensions, da sie den Namen der Hauptmarke enthalten und auch deren Image mit übernehmen, jedoch in einen anderen Produkt-Sektor gehören. Würden sie als komplett neue Marke verkauft werden, wäre dies eine Brand Extension.


\section{Aufgabe 2}
\subsection{2.a}

\subsection{2.b}

\subsection{2.c}

\subsection{2.d}

\subsection{2.e}

\subsection{2.f}


\section{Aufgabe 3}
\subsection{3.a} 
\subsection{3.b}

\subsection{3.c}

\subsection{3.d}
\subsection{3.e}

\subsection{3.f}

\end{document}