

%opening
\title{Sommersemester 2019 Nachklausur}
\maketitle



\section{Aufgabe 1}
\subsection{1.a}
    symbolischer Nutzen: Voss verkauft sich als Luxus-Lifestyle-Marke wie Apple oder Mercedes. \\
    Reduktion des Risikos: Das wahrgenommene Risiko ein minderweertiges Proidukt zu kaufen wird durch die Marke verringert.

\subsection{1.b}
    %TODO Grafik
    Voss: Oberpreissegment
    GErolsteiner: Mittelpreissegment
    Europerl: Niedrigpreissegment
    Leitungswasser: Niedrigpreissegment

\subsection{1.c}
    Point Of Difference: Punkte, in denen sich das Produkt von denen der Konkurrenz unterscheidet \\
    Points Of Parity: Punkte, in denen sich das Produkt von denen der Konkurrenz nicht unterscheidet. \\
    \ \\
    Ein Point of Parity für Karlsruher Wasser im Vergleich zu Voss könnte die (angebliche) Nutzung von Gletscherwasser sein. \\
    Es ist wichtig beide Punkte zu betonen, damit man als Kunde den Eindruck bekommt dass das Produkt alles kann was die Konkurrenz auch kann (man "verpasst" also nichts) und sogar noch mehr. \\

\subsection{1.d}
    Punkt 1: Herkunftsland Deutschland. Desirable, because german products are of high quality. Deliverable, because the water comes from germany. Differentiation, because Voss has explicitly stated that their water comes from another country. \\
    Punkt 2: Double Filtration. Desirable, because filtering something is always good. Deliverable, because water can be filtered. Differentiation, beaucase most water is not double filtered. \\
    %Hier hab ich echt kp was ich schreiben soll -- worin unterscheidet sich wasser???????? ist doch alles das selbe zeug
    


\section{Aufgabe 2}
\subsection{2.a}
    Das Preisimage ist die Einschätzung des Preisniveaus eines Einkaufsortes. \\
    Das Preisniveau ist für Amazon Go relevant, da die Konkurrenz im Einzelhandel sehr groß ist und ohne zusätzlichen Service (Personal a.d. Kassen) der Preis das wichtigste Merkmal ist. \\

\subsection{2.b}
    Mit dynamic pricing ist die Preisdifferenzierung für jeden Kunden einzeln anhand im Netz erfasster Daten und Attribute gemeint. \\
    Es gehört damit zur Differenzierung ersten Grades, da jeder Kunde individuell betrachtet wird.

\subsection{2.c}
    Es gibt das Problem, dass manche Kunden mitbekommen wie andere ein identisches Produkt zu unterschiedlichen Preisen kriegen. \\
    Die Kunden können das durch Weiterverkauf der Waren zu ihrem Vorteil ausnutzen. Waren werden günstig durch eine Person gekauft und dann mit leichtem Aufschlag etwas teurer weiterverkauft.

\subsection{2.d}
    \begin{itemize}
        \item Segmentierung: Durch die App einfach Möglich, jeder Kunde kann anhand der vorher gesammelten Daten schon vor dem 1. Einkauf in sein Segment eingeordnet werden.
                Problematisch wird es nur bei Leuten, zu denen es kaum bis keine Daten gibt.
        \item FAirness: Nicht gegeben. Da man direkt mit anderen einkauft kann man sehr schnell die Preise vergleichen. Gerade bei Lebensmitteln will auch keiner mehr als andere ausgeben. \\
                Auf der anderen Seite sind die Preise nicht an den Produkten befestigt, was die Vergleiche schon etwas schwerer macht.
    \end{itemize}

\subsection{2.e}
    Leistungsbezogene Differenzierung: So können Kunden die ein Prime-Abo haben, den Laden länger nutzen oder schneller zahlen (Expresskassen)
    Zeitliche Differenzierung: Preise für Alkohol könnten gegen Abend angehoben werden, Preise für Salate zB mittags



\section{Aufgabe 3}
\subsection{3.a}
    Die Einstellung gegenüber Frauenfussball nimmt die periphere Route, da die ZUschauer einfach von der Leistung der dt. Mannschaft überzeuigt werden können. Das Involvement für Frauenfussball ist allgemein gering. \\
    Die Commerzbank nutzt dabei die zentrale Route und nutzt die positive Einstellung zur Kommunikationsmaßnahme, um eine emotionale Verbindung herzuustellen.
    %TODO kann auch genau anders herum sein. oder beide zentral???

\subsection{3.b}
    Affektiv. Die Werbung nutzt die positiven Gefühle des Kampfes der Frauen um Anerkennung, damit diese Gefühle auf die Bank übertragen werden.

\subsection{3.c}
    Mit viraler Werbung ist eine virusartige schnelle Ausbreitung der Werbung und seiner Nachricht (zB dem Slogan) gemeint.

\subsection{3.d}
    Präzision meint die genaue Ansteuerung der Werbung auf die Zielgruppe. Der Streuverlust sind dabei Menschen die die Werbung zwar sehen, aber sich nicht angesprochen fühlen oder die Werbung gar nicht verstehen. \\
    \ \\
    Eine virale Kampagne hat immer einen hohen Streuverlust, da viele zwar die Botschaft sehen werden aber nicht jeder sie mit der Bank assoziieren wird. Gerade bei Neukunden kann diese Assoziation schnell wegfallen, da viele wahrscheinlich nicht so oft über ihre Bank nachdenken. \\
    Ausserdem gibt es viele Skeptiker was Frauenfussball anageht, zusammen mit einer großen Menge an Vorurteilen. \\
    Diese werden die Werbung niucht lustig finden und ggf. sogar negativ beeinflusst werden. \\

\subsection{3.e}
    %TODO Attitude towards the ad erklärung/verstehen
    Ein alternatives Instrument um die Wirkung von Werbung zu messen ist die Nutzung von Gutscheinen oder Rabatten in der Verbindung mit der Werbung.

\subsection{3.f}
    JA, da die Aufmerksamkeit bei Werbevideos vor Onlinevideos extrem gering ist.