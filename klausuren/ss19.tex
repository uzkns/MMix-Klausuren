\documentclass{article}
\usepackage[ngerman]{babel}
\usepackage[utf8]{inputenc}

\title{Sommersemester 2019 Hauptklausur}
\begin{document}
\maketitle


\section{Aufgabe 1}
\subsection{1.a}
\subsubsection{symbolischer Nutzen}
Voss steht für reines Gletscherwasser und hohe Wasserqualität ein. Mit dem Kauf der Wassers gönnt man sich nicht nur etwas überlebenswichtiges, sondern zusätzlich auch noch ein Luxusprodukt. Dies zieht natürlich Kunden an, die ihren Status gerne in der Öffentlich preis geben.
\subsubsection{emotionaler Nutzen}
Da verschiedene bekannt Stars als Testimonials für Voss einstehen, macht sich Voss die emotionale Bindung von Verbrauchern zu Nutzen. Leute, die sich mit den für Voss werbenden Stars identifizieren, würden sich demnach zu diesem Wasser mehr hingezogen fühlen.

\subsection{1.b}
Sowas wird wohl eher nicht dran kommen.

\subsection{1.c} 
Point of Difference = Unterschied zum Konkurrenten.\\
Point of Parity = Ähnlichkeit/Gleicheit zum Konkurrenten.\\
Ein Point of Parity könnte die hohe Wasserqualität des Karlsruher Wassers sein, wichtig wäre dabei aber zu betonen, dass man nicht wie Voss versucht, etwas anderes zu promoten, sondern das verkauft, was man auch anpreist.

\subsection{1.d}
\subsubsection{"`Es ist drin, was drauf steht"'}
Desirability: Leitungswasser ist ein im Alltag sehr nützliches Produkt. Kunden brauchen dieses also durchaus. \\
Deliverability: zB durch Design mit Karlsruher Fächer-Emblem oder Wappen überbringbar. Ein Problem könnte dabei auftreten, zu kommunizieren, warum das Leitungswasser aus Karlsruhe gekauft werden soll, statt das eigene aus dem Hahn zu verwenden.\\
Differentiation: Hier kann mit Ehrlichkeit geworben werden i Unterschied zu Voss.\\

\subsubsection{Mittelpreiswasser statt Luxusgut}
Desirability: Die dadurch angesprochene andere Zielgruppe verlangt nach Wasser nicht als Luxusgut und kann somit durch geringere Preise überzeugt werden.\\
Deliverability: Kann durch ein angepasstes Design, zB durch die Verwendung von Mehrwegpfandflaschen statt Designerflaschen erzeugt werden.\\
Differentiation: Durch ein anderes Flaschendesign ist der Unterschied zu Voss deutlich erkennbar, eine Herausforderung ist es aber, zwischen anderen Mittelpreisigen Wassern hervorzustechen.\\

\section{Aufgabe 2}
\subsection{2.a}
Das Preisimage beinhaltet die Bewertungen eines Käufers gegenüber dem Preisniveau des Anbieters. Diese Bewertung erstellt der Käufer anhand Stichprobenartiger Bewertungen von einzelnen Produkten aus dem Sortiment des Anbieters.
Wenig Service --> einziges was zählt ist Preis.
Gleiche Preise wie online --> Kunden haben Vertrauen, da sie wissen, was auf sie zukommt.

\subsection{2.b}
Beim Dynamic Pricing werden jedem Kunden unterschiedliche Preise angezeigt. Diese werden anhand verschiedener Daten festgemacht, zB welches Handy-Betriebssystem der Kunde hat, oder aus welcher Gegend er kommt. Dynamic Pricing ist Preisdifferenzierung 1. Grades, da jeder Kunde ein persönlicher Preis erhält.

\subsection{2.c}
Kunden können sich als unfair behandelt betrachten und deswegen kann ihre Zufriedenheit sinken. Kunden können dieses System außerdem opportunistisch ausnutzen, indem sie sich zusammentun und nur noch über den "`günstigsten"' Account kaufen.

\subsection{2.d}
\subsubsection{Möglichkeit zur Kundensegmentierung}
Ist bei Amazon Go gut möglich. 

\subsubsection{Wahrnehmung als fair}
Durch die Preisdifferenzierung 1. Grades sind, wie in c) erläutert, Probleme bei der Wahrnehmung als fair vorhanden, sofern sich Kunden austauschen.

\subsection{2.e}
\subsubsection{Räumliche}
Aufgrund von geografischer Einordnung verschiedene Produkte im Amazon Go Sortiment anbieten.
\subsubsection{Mengenbezogene}
Durchschnittl. Preis pro Produkt variiert, es können unter anderem Flatrates für gewisse Produkte (zB Druckertinte) gekauft und somit Rabatte erzielt werden.

\section{Aufgabe 3}
\subsection{3.a} 
\subsubsection{Frauenfußball}
zentrale Route: emotionale Ansprache der Errungenschaften der Frauenelf erzielt beim Beobachter eine Identifikation und Stolz mit der Mannschaft. Die Motivation sich zu informieren steigt.
\subsubsection{Commerzbank}
periphere Route: Commerzbank Logo nur am Ende kurz gezeigt --> kein hohes Involvement bei Beobachter erzielt

\subsection{3.b}
affektiv: es wird gezielt angesprochen, was die Fußballerinnen positives geschafft haben. Damit wird der Beobachter emotional beeinflusst.

\subsection{3.c}
Verbreitung über das Internet und Online-Marketing. Es wird unter anderem darauf gesetzt, dass die Werbung "`weitergezeigt"' wird.

\subsection{3.d}


\subsection{3.e}


\subsubsection{3.f}


\end{document}