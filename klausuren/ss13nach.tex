

%opening
\title{Sommersemester 2013 Nachklausur}
\maketitle


\section{Aufgabe 1}
\subsection{1.abcd}
    Bassmodell kann nicht gefunden werden

\subsection{1.e}
    \begin{itemize}
        \item Dachmarke
        \item Markenfamilie
        \item Einzelmarken
    \end{itemize}

    \begin{itemize}
        \item Positiv: Höhere Bekanntheit der Marke durch verstärktes Auftreten
        \item Positiv: Goodwill wirkt sich auf alle Produkte aus
        \item Positiv: Designs von Verpackungen können zT wiederverwendet werden (Resourcenschonend)
        \item Negativ: Badwill-Transfer
        \item Negativ: Produkte können nur eingeschränkt spezifisch profiliert werden
    \end{itemize}

\subsection{1.f}
    Produktdiffusion??

\section{Aufgabe 2}
\subsection{2.a}
    SMART-Kriterien??

\subsection{2.b}
    Ergebnisziele?? Verhaltensziele??


\section{Aufgabe 3}
\subsection{3.a}
    Kosten-Plus-Pricing startet mit den Herstellungskosten und addiert auf diese alle sonstigen entstandenen Kosten und den geplanten Gewinn obendrauf. \\
    Mit der geplanten Absatzmenge wird damit der Preis pro Einheit festgelegt.

\subsection{3.b}
    Kosten-plus-Pricing hat keine Möglichkeit die Herstellungs- und Forschungskosten zu reduzieren. \\
    Diese Kosten werden als gegeben angesehen, was dazu führen kann dass die Produkte am Ende "over-engineered" und zu teuer sind.

\subsection{3.c}
    Wir haben Target Costing statt Value-in-Use

\subsection{3.d}
    Der Preis als Qualitätsindikator spielt hierbei eine große Rolle. Da Unternehmen in ihre einkäufe ein hohes Involvement haben sind die Preise im Markt sehr kompetitiv. \\
    Ein höherer Preis deutet also direkt auf eine höhere Qualität hin, da weniger gewartet werden muss bzw das Produkt länger hält.


\section{Aufgabe 4}
\subsection{4.a}
    Mass-Food GmbH: Für die Schokobiene würde ich TV-Werbung vorschlagen. \\
    Es werden viele Menschen angesprochen und mit dem richtigern TV-Sender kann die Zielgruppe (Kinder) einigermaßen gut gesteuert werden. \\
    Print- und Onlinewerbung bieten sich nicht an, da sie zu wenig Sinne stimulieren. \\
    Bei einem Essensprodukt ist es wichtig, nicht nur das Produkt zu sehen. Es muss auch ein bestimmtes Gefühl, zB durch Musik vermittelt werden. \\
    \ \\
    \ \\
    Cemento AG: Für den Zement würde ich Printwerbung nutzen. \\
    Der MArkt für Zement ist sehr klein, es müssen generell nicht viele Leute angesprochen werden. Die Allgemeinheit interessiert sich nicht für Zement, die Zielgruppe sind nur Bauleiter. \\
    Durch Druck in einer Fachzeitschrift kann diese Zielgruppe sehr genau angesprochen werden. \\
    Da der Verkauf hauptsächlich B2B ist, hat Zement gar keinen symbolischen oder emotionalen Nutzen. \\
    Dem Unternehmen muss kein Gefühl vermittelt werden, es zählen nur die Fakten. Diese können im Print sehr ordentlich präsentiert werden. \\
    \ \\
    \ \\
    PerCom AG: Onlinewerbung passt am besten. \\
    Die Zielgruppe sind Menschen, die sowieso das Internet benutzen. Durch Onlinewerbung wird diese Zielgruppe perfekt getroffen. \\
    Durch die Möglichkeit, animierte Bilder und kurze Clips in die Onlinewerbung aufzunehmen wird es möglich, die Personlisierungmöglichektien (zB verschiedene Farben) gut darzustellen. \\
    Duch einen Button mit "Jetzt bestellen" wird der Kunde direkt zur Bestellseite weitergeleitet, wo der Kunde direkt Informationen zum Bestellporzess findet. \\

\subsection{4.b}
    Gesundheitsverprechen sind dem Markennutzen zuzuordnen. Sie versprechen dem Nutzer einen gesundheitlichen Vorteil beim Nutzen der Marke ggü. anderen Marken, zum Beispiel durch die Menge an Zucker oder die Verwendung von Bio-Zutaten. \\
    Den meisten Kunden wird klar sein dass Schokolade kein gesundes Produkt ist, deswegen wirkt das Versprechen von Gesundheit negativ. Eine Schokoladenmakre wirkt nicht glaubwürdig wenn es um Gesundheit geht.