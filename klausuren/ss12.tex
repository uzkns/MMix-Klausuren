

%opening
\title{Sommersemester 2012 Hauptklausur}
\maketitle

\section*{Aufgabe 1}
\subsection*{1.a}
    \begin{itemize}
        \item Einsparung an Informationskosten
        \item emotionaler Nutzen
        \item Reduktion des Risikos
        \item Symbolischer Nutzen
    \end{itemize}

    Der emotionale Nutzen ist von den Kategorien am wichtigsten. \\
    Da es einen starken Wettkampf gibt ist für den Kunden das Risiko allgemein gering. \\
    Die Informationskosten beim Wodkakauf sind auch eher gering und Wodka ist kein "Lifestyle", also ein geringer symbolischer Nutzen.
\subsection*{1.b}
Von innen nach außen:
    \begin{itemize}
        \item Markenkern
        \item Markennutzen - Der Nutzen, genau dieses Produkt zu kaufen und nicht das der Konkurrenz:
        \begin{itemize}
            \item Guter Geschmack
            \item Diamantfiltriert, besonders rein
            \item 1L Flasche für mehr Inhalt
        \end{itemize}
        \item Markenpersönlichkeit - Menschliche Attribute der Marke
        \begin{itemize}
            \item lebhaft
            \item gesellig
            \item spritzig
        \end{itemize}
    \end{itemize}
\subsection*{1.c}
%\includegraphics[width=\textwidth]{klausuren/res/ss12-1c.png}
Das ELM hat 2 Routen, um die Einstellung der Kunden zu ändern: Die zentrale Route und die periphäre Route: Bei einer hohen Einstellung der Kunden (hohe Motivation, hohe Fähigkeit, hohe Gelegenheit) wird die zentrale Route gewählt. Bei niedriger Einstellung die periphäre Route. \\
Für den Wodka ist die periphäre Route besser geeignet, da die Motivation, sich außgiebig mit einem Wodka zu beschäftigen sehr gering ist. Ein bloßer Kontakt mit der Marke reicht aus, damit sie in Erwägung gezogen wird.

\subsection*{1.d}
Das Risiko bei der Nutzung von Humor in der Werbung ist, dass die Leute es auch witzig finden müssen. Ansonsten wird die Werbemaßnahme nur als "nervig" gesehen. Dazu gehört auch, dass der Humor auf die Zielgruppe abgestimmt werden muss und nicht alle Gruppen den gleichen Humor haben. \\
Der Witz muss also auch verstanden werden, ansonsten ist er absolut nicht lustig. \\
Da die Zielgruppe eines Wodka junge Erwachsene mit Drang zu Gesellschaft und Party sind, passt Humor gut zu der Zielgruppe.

\subsection*{1.e}
\begin{itemize}
    \item Fit zwischen Testimonial und Marke - Das Testimonial muss klar erkennbar zur Marke passen
    \begin{enumerate}
        \item Biathlet
        \item Bohlen
        \item Norge
    \end{enumerate}
    \item Fit zwischen Testimonial und Zielgruppe - Das Tesitmonial muss ind er Zielgruppe bekannt sein oder ansehen haben
    \begin{enumerate}
        \item Bohlen
        \item Biathlet
        \item Norge
    \end{enumerate}
    \item Anzahl der Produkte, die das Testimonial bewirbt (abst.) - Das Testimonial darf nicht zu viele Produkte bewerben, da sonst die Glaubwüridgkeit sinkt
    \begin{enumerate}
        \item Norge
        \item Biathlet
        \item Bohlen
    \end{enumerate}
\end{itemize}

Ich würde Werbung mit dem Biathleten machen. \\
Das Maskottchen ist nur sehr schwer mit der Zielgruppe und dem Produkt vereinbar und Dieter Bohlen macht für sehr viele Produkte Werbung, er ist also nicht besonders glaubwürdig. \\
Der Biathlet ist zwar in Deutschland nicht sonderlich bekannt, aber er kann Norwegen gut vertreten, passt also zur Marke. Ausserdem macht er kaum/keine Werbung in DE, wirkt also sehr glaubhaft. 

\subsection*{1.f}
Preisdifferenzierung ist die Mögichkeit, sein Produkt verschiedenen Zielgruppen zu unterschiedlichen Preisen anzubieten. Dadsurch werden die Zahlungsbereitschaften optimal ausgereizt. \\
\\
Bei Wodka ist die Konkurrenz sehr stark und das Produkt kann schnell übertragen werden indem man es an einem "günstigeren Ort" kauft. Es würde also nicht helfen, das Produkt zu drastisch unterschiedlichen Preisen anzubieten. \\
\begin{itemize}
    \item Personenbez. Segmentierung - Das Produkt verschiedenen Kunden zu verschiedenen Preisen anzubieten ist hierbei schwer, da Alkohol hauptsächlich über den Einzelhandel verkauft wird und diese nicht nach Kundenmerkmalen unterscheidet
    \item Absatzmenge - Ein Mengenrabatt ist möglich (zB für Großbabnehmer wie Clubs, Bars, etc.) aber im Einzelhandel kaum durchsetzbar.
    \item Räumliche Segmentierung - Gut möglich, da Alkohol meist "vor Ort" recht spontan gekauft wird. Allerdings muss auch hier mit Grauimporten gerechnet werden
    \item Zeitliche Segmentierung - Gut möglich, da an Wochenenden idR mehr Zusammenkünfte stattfinden als unter der Woche kann der Preis angepasst werden.
\end{itemize}

\section*{Aufgabe 2}
Ich habe den morphologischen Kasten nicht gefunden. \\
Ich glaube das Thema ist im Sommersem. 2020 nicht mehr relevant.
\subsection*{2.a}
\subsection*{2.b}
\subsection*{2.c}
Kosten-Plus-Preisbildung fängt an mit den Produktionskosten und addiert die dazu entstandenen Kosten (zB durch Werbung, Vertrieb, Logisitik etc.) darauf. Dadurch werden die gesamten Kosten eines Produktes berechnet. \\
Diese Kosten werden geteilt durch die angestrebte Absatzmenge und so entsteht der Preis für eine Einheit. \\
\ \\
Beim Target Costing sind die Gesamtkosten und der angestrebte Gewinn gegeben. Das Marketing wird durch Sparmaßnahmen so angepasst dass diese Kosten eingehalten werden. Würden die Kosten nicht eingehalten, so wäre der Gewinn geschmälert und der Preis müsste erhöht werden.\\
\ \\
Bei Einbauküchen würde man eine Küche für 1999€ anbieten und dann gucken wo man sparen kann um auf diesen Preis zu kommen (zB Lieferung, Montage, Werbung, Materialien)

\section*{Aufgabe 3}
\subsection*{3.a}
Gleitender Durchschnitt $n$-ter Ordnung ist der Durchschnitt der letzten $n$ Werte. (Hier: $n=3$) \\
\ \\
$(13400+18800+22000)/3 = 18067$ Stk. \\

\subsection*{3.b}
Die Methode nimmt keine Veränderungen an, d.h. der Absatz wird nicht durch Faktoren gestört, die es vorher nicht gab (zB ein neues Konkurrenzprodukt). \\
Ein kleines $q$ ist zu stark anfällig für kurze Trends (zB ein außerordentlich gutes Verkaufsjahr), ein großes $q$ glättet diese Trends ggf. zu stark. Wenn der Absatz in den letzten Jahren konstant stark zunimmt, aber vorher gering war, kann ein zu großes $q$ einen zu geringen Wert liefern. \\

\subsection*{3.c}
Das Modell der Zielsetzung ist mir nicht bekannt(?) \\

\subsection*{3.d}
Die Stückkosten bestimmen den Preis (mehr Kosten = Produkt muss teurer verkauft werden), welcher den Absatz bestimmt (geringerer Preis = mehr Absatz). Der Absatz hat wieder auswirkungen auf die Stückkosten (mehr Absatz = mehr produzierte Waren = geringere Stückkosten). Dies spiegelt sich dann in einer Verringerung des Preises wieder, womit der Kreis geschlossen ist. \\
\ \\
Das logische Problem hierbei ist, dass eine Veränderung einer Variable (Kosten, Preis, Absatz) sich nach und nach auch auf alle anderen Variablen auswirkt. \\
Eine Überschätzung des Absatzes ist damit sehr schlecht, da die rel. Stückkosten bei Produktion vieler Produkte zwar sinken, aber die absoluten Kosten weiter steigen. \\
Wird jetzt nur ein Teil der Waren verkauft, so hat man Produkte die zwar Unkosten verursacht haben aber keine Gewinne einbringen. Un die geringen Preise der verkauften Produkte können die absoluten Kosten nicht ausgleichen.