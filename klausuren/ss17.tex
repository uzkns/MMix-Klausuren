

%opening
\title{Sommersemester 2017 Hauptklausur}
\maketitle



\section{Aufgabe 1}
\subsection{1.abcdef}
    %THEMA Komplexitöätskosten, Opportunitätskosten, Verschiebung der Komplexität auf spätere Wertschöpfungsstufen



\section{Aufgabe 2}
\subsection{2.a}
    %TODO strategische bias hypothetischer bias
    Der strategische Bias ist die Überschätzung des Produktes und der damit verbundenen ZB. \\
    Der hypothetische Bias ist die Verzerrung der ZB durch die Befragten, da man in der Befragung das Produkt nicht wirklich kaufen muss.

\subsection{2.b}
    Der hypothetische Bias geht bei der Conjoint-Analyse herunter, da diese das Kaufverhalten durch die Merkmale und Ausprägungen besser darstellt. \\
    Ganz verschwinden tut er nicht, da auch hier wieder ein Unterschied zwischen wahren Kaufverhalten und angegebenem Kaufverhalten stattfindet.

\subsection{2.c}
    \begin{itemize}
        \item Sensorgröße: APSC oder 35er Vollformat
        \item Auflösung: 36MP, 20MP, 24MP
        \item AF-Pkte.: 51, 33, 45
        \item Bilder/sek.: 8, 4, 6
        \item Anbindung: Wifi, Wifi+BT, Wifi+BT+NFC
        \item Preis: 1000€/1500€/2000€, 2200€, 900€
    \end{itemize}

\subsection{2.d}
    Bei allen Merkmalen sind die Konkurrenzprodukte abbildbar.
    Jedoch hat ein Vollformatsensor standardmäßig eine höhere Auflösung als ein APSC, diese beiden Merkmale sind also gänzlich nicht unabhängig voneinander.

\subsection{2.e}
    %THEMA malen

\subsection{2.f}
    Gesamtnutzen Gewicht: 4,5
    %TODO

\subsection{2.g}
    %THEMA Luce Modell



\section{Aufgabe 3}
\subsection{3.a}
    Die Platzierung im Super Bowl-Finale ist für die Kontakte nicht zu übertreffen: Der Superbowl ist das größte Event im amerikanischen Fernsehjahr. \\
    GTleichzeitig sinkt dadurhc auch die Präzision: Sehr viele unterscheidliche Menschen gucken den Super Bowl, dadurch auch ein großer Teil der nicht Teil der Zielgruppe ist. \\
    Laut Text zielt die Werbung auf Frauen ab, der Streuverlust ist also auf den ersten Blick sehr hoch.
    Da der Super Bowl ein Sportevent ist, ist die erreicht Gruppe wahrscheinlich hauptsächlich männlich. \\
    \ \\
    Diese Streuverluste sind irrelevant, da für das Produkt Männer die Zielgruppe sind, auch wenn die Werbung anscheinend auf Frauen ausgelegt ist. \\

\subsection{3.b}
    Shampoo ist für mich auf jeden Fall ein low involvement Produkt. Die Nutzung von Huimor macht hier Sinn. \\
    Andererseits ist die Zielgruppe von Old Spice nicht auf junge Menschen ausgelegt, der Bildungsstand ist hierbei auch irrelevant. Der Humor passt nach diesem Merkmal also nicht so gut\\
    Humor passt auch zu den meisten Zielgruppen also wird es wohl auch auf die Kultur der Männer passen. (was ist das für eine frage bitte was ist die kultur deiner zielgruppe wenn deine zielgruppe FÜNFZIG PRIZENT ALLER LEBENDEN MENSCHEN IS ???)

\subsection{3.c}
    %THEMA Sex als Verkaufsargument

\subsection{3.d}
    Es wurde der Aided Recall abgefragt. \\
    Eine andere Methode ist die Abfrage der Dominanz, also ob die Marke die einzig bekannte Shampoo-Marke ist.

\subsection{3.e}
    %THEMA STAS Differenzial



\section{}