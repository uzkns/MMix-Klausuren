

%opening
\title{Sommersemester 2019 Nachklausur}
\maketitle


\section{Aufgabe 1}
\subsection{1.a}
    Das Modell der VL bezieht sich auf die Veränderung in Preis und Absatz wenn eine Marke dazu kommt. \\
    Dabei wird der Preis eines Produktes mit/ohne Marke verglichen und der Absatz eines Produktzes mit/ohne Marke verglichen. \\
    Aus der Differenz setzt sich der Markenwert zusammen.

\subsection{1.b}
    Die Marke "Amazon" hat einen höheren Markenwert als die Umsätze es suggerieren, da der Anteil der Marke zwar nicht viel am Umsatz geändert hat, aber die Kundenwahrnehmung der Marke stark gestiegen ist. \\
    Amazon erreicht dies durch stärkere Nutzung von Eigenmarken und dem Ausbau der Marke (Line extension).

\subsection{1.c}
    Das Kudnenwissen muss gleichzeitig spezifisch abgefragt und holistisch ermittelt werden. \\
    Dabei ist auch das oft nur implizite Markenwissen eine Herausforderung.
    %TODO Lösung

\subsection{1.d}
    Markenidentität ist der Eindruck der Marke, festgelegt anhand von Attributen und dem Auftreten. \\
    Amazon sieht sich als freundliches Unternehmen, dewegen ist ein lächeln im Logo vertreten.\\
    \ \\
    Markenerlebnis ist das Ziel, über alle Touchpoint gleich auftritt, man die Marke also immer gleich "erlebt". \\
    Das wird erreicht durch einen konsistenten Marketing Mix, bei Amazon zum Beispiel durch Werbung oder die preisliche einsortierung. \\

\subsection{1.e}
    Ein Konsistenter Markenauftritt ist sehr wichtig, damit man als Kunde direkt die gemerkten guten Assoziationen auf andere Produkte des Unternhemens überträgt und direkt weiß von wem dieses Produkt statt und was man davon erwarten kann. \\
    Dabei werden aber auch Badwill-Transfers ermöglich, also die Transferierung von negativen Assoziationen auf andere Produkte.

\subsection{1.f}
    Bei der Line extension werden neue Produkte zusammen mit bisherigen Produkten unter der selben Marke angeboten. So ist Amazon Music nur ein neues Produkt der Marke "Amazon".\\
    Bei der Brand Extension wird die Marke um weitere ähnliche amrken erweitert, unter welcher dann die neuen Produkte nagenboten werden. So ist Prime Video eine Marke welche ähnlich der Stammmarke Amazon ist (Prime aus dem Versand, Lächeln im Logo).
    Die alten Produkte werden dann weiterhin unter der alten Marken angeboten.


\section{Aufgabe 2}
\subsection{2.a}
    Die Kosten-Plus_preisbildung folgt dem Prinzip: Wie viel soll das Produkt kosten? Und rechnet den Stückpreis aus, indem die Herstellungs- und Marketingkosten mit dem geplanten Gewinn addiert werden. \\
    Beim Target Costing gibt es schon einen festen Maximalpreis. Die Herstellungs- und Marketingkosten werden so angepasst, das bei diesem Preis noch Gewinn erzielt wird. \\
    \ \\
    Beides ist sehr schwer, da eine App keine Produktionskosten pro Stück hat und der Absatz nicht vorher festgelegt werden kann. \\

\subsection{2.b}
    Die direkte Kudnebefragung ist für die Migräne App gut geeignet. Aufgabe 2.a hat gezeigt dass es sehr schwer ist einen Preis für die App zu ermitteln. Durch dir Kundenbefragung wird der Preis in den Vordergrund gerückt, also wird das Preisbeusstsein sehr hoch sein. \\
    Problem ist dabei, dass VAriable Menge-Fragen nicht genutzt werden können und die Kunden von ihrem angegebenen Verhalten abweichen werden.

\subsection{2.c}
    Es handelt sich um personenbasierte Segmentierung. Jedem Kunden wird ein individueller Preis für die selbe Leistung angeboten. \\
    Der Algorithmus ermittelt damit die Merkmale der Kunden.

\subsection{2.d}
    \begin{enumerate}
        \item Monopol: Gegeben, da die Konkurrenzprodukte sehr geringer Nutzerzahlen aufweisen
        \item Nichtübertragbarkeit: Gegeben, da die App durch die Daten personalisiert wird. Ausserdem können IAP nicht weiter gegeben werden.
        \item Fair: Ggf. gegeben, hängt von der Einstellung des Algorithmus under der Einstellung der kunden ab
        \item Segmentierungmöglichekti: Durch den Algorithmus gegeben
        \item Verhältnismäßigkeit: wahrscheinlich nicht gegeben, App-Preise sind sehr gering und einen Algorithmus zu entwickeln ist nicht günstig.
    \end{enumerate}

\subsection{2.e}
    Die Grundidee der nutzenorientierten Preisbestimmung ist es, die für Kunden relevantesten Merkmale rauszufinden und auf Basis dieser Merkmale und einem Vergleichsprodukt den Preis zu bestimmen. \\
    Zwei Voraussetzungen sind ein Vergleichsobjekt/eine Referenz, welche hier aufgrund mangelnder Konkurrenz nicht gegeben ist und die Möglichkeit, Merkmale festzusetzen und ihre Ausprägung festzustellen. Dies ist hier gut gegeben.


\section{Aufgabe 3}
\subsection{3.a}
    Das Ziel des ELM ist es, die Auswirkung von Werbung auf die Einstellung von Kunden in 4 Wege grundsätzlich auf Basis von Involvement einzuteilen: \\
    Die zentrale route für hohes Involvement und die periphäre route für niedriges Involvement. \\
    Dazu kommt noch die Einteilung in die Grundlage der Einstellung: Affekt oder Koginition. \\
    \ \\
    Die Doir-Werbung wirkt auf der peripheren Route und arbeitet auf Grundlage von Kognitionen: Durch den gezeigten Star und Glamour wird eine einfache Überzeugung, dass Dior hochweertig ist und einen schön macht, erzeugt. \\

\subsection{3.b}
    Sex ist sexyy -> gut für den verkauf
    %TODO keine ahnung... sex ist immer gut dachte ich

\subsection{3.c}
    \begin{itemize}
        \item Aufmerksamkeit ist durch Werbepausen eher gering. Viele Leute gehen auch gerade dann auf die Toilette oder gucken zeitversetzt fern.
        \item Sinne: Es werden die audiovisellen Sinne angesprochen
        \item Handlungsbezug: Man sitzt vor dem Fernseher. Ohne aufzustehen, rauszugehen und einzukaufen ist kein Handlunsgbezug gegeben.
    \end{itemize}

    Die Dior-Werbung ist nicht besonders gut geeignet. Durch die MAsse an Beauty-Werbung geht die Werbung unter und ohne Handkungsbezug wird das Produkt nicht lange in den Köpfen der menschen bleiben.

\subsection{3.d}
    Das Fit zwischen Testimonial und Zielgruppe ist wie sich die Zielgruppe mit dem Testimonial identifizieren kann. \\
    Bei Charlize Theron ist dieses durch das Alter gegeben. Allerdings wird sich nicht jeder mit ihr idetifizieren können oder wollen. \\
    Alles in allem denke ich aber passt sie als Testimonial für eine Beautymarke sehr gut.

\subsection{3.e}
    %TODO Keine Ahnung
    \begin{itemize}
        \item "Nennen sie den NAmen dieser Marke"
        \item "Nennen sie die sechs bekanntesten Beautymarken"
        \item "Wie heißt die Werbeperson von Dior"
    \end{itemize}

    Ein weitere Methode wäre die Auswertung von Social Media Indikatoren