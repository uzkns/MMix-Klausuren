

%opening
\title{Sommersemester 2019 Nachklausur}
\maketitle



\section{Aufgabe 1}
\subsection{1.a}
    Herr Bender bezieht sich auf den emotionalen Nutzen der Marke. Er verbindet die Marke Lufthansa und ihr Logo bzw. ihre Farbwahl mit Attributen wie souverän und vertrauenswüridg.

\subsection{1.b}
    Das sensorische Gedächtnis ist für die die Aufzeichnung von Sinneseindrücken verantowrtlich (zB die Farbe gelb hinterlässt einen warmen, freundlichen eindruck und die Interkation mit dem Flugpersonal lässt die Mitarbeiter sehr meschlich und freundlich wirken). Durch einen Lernproizess werden diese Eindrücke ins Kurzzeitgedächtnis übertragen, wo sie geübt werden. \\
    Am Endse werden die Informationen kodiert und die Assoziation (Lufthansa <-> freundlich) wird ins Langzeitgedächtnis übertragen.

\subsection{1.c}
    Die Markenidentität ist die Assoziation von menschlkichen Merkmalen mit einer MArke. Für Lufthansa ist die Identität wichtig, da sie als hochpreisige Airline unbedingt ein bestimmtes Image verkaufen muss, um sich von der Konkurrenz abzusetzen.

\subsection{1.d}
    Die DAchmarkenstrategie ist die Nutzung einer Marke für alle Produkte. \\
    Für ihre Tochtergesellschaften verwendet Lufthansa eigene Logos, da diese meist in einem anderen Preissegement liegen und durch Einzelmarken die Ansprache verscheiender Segemnte erleichtert wird. \\
    Gleichzeitig gibt es bei einer DAchmarke einen starken Goodwill-/Badwill-Transfer auf andere Produkten. Bei Einzelmarken ist dies sehr schwer. Für die Lufthansa beduetet dies dass das Image der Tochtergesellschaft nicht vom Image der Lufthansa abhängt. \\

\subsection{1.e}
    Es wurde der Wiedererkennungswert des Logos abgefragt.

\subsection{1.f}
    Da viele Leute das alte Logo kennen und sich die Form und Inhalt nicht geändert haben, werden auch viele das neue Logo erkennen. \\
    Allerdings kann es vorkommen, dass jemand sich am ehesten noch an die Farbe der LH errinnern kann. So jemand würde dann nach etwas gelben ausschau halten und würde damit das neue Logo nicht erkennen. \\
    So konnte man zum Beiospiel an einem überfüllten Flughafen noch irgendwo etwas gelbes sehen und sich daran orinitieren. Das geht jetzt nicht mehr. 



\section{Aufgabe 2}
\subsection{2.a}
    Es handelt sich um gemischstes Bundling. Beim Bundling werden mehrere Produkte zusammen verkauft (in einem sog. "Bundle"), bei der Spzialform des gemischten Bundles können diese Produkte auch einzeln erworben werden. \\

\subsection{2.b}
    Adobe will durch das Bundle erreichen, das man ggf. Software dazukauft, die man vorher nicht brauchte, wollte oder gar kannte. \\
    Die Adobe CC erreicht diese Ziel dadurch, dass es extreme Preisvorteile gibt, wenn man die Produkte im Bundle kauft und schon ab wenigen benötigten Produkten es Sinn macht, ein Abo über alle Apps abzuschließen.
    
\subsection{2.c}
    Sie würde nur InDesign kaufen. (gute wahl illustrator suckt nämlich HART im verlgiehc zu inkscape) \\
    Da ihre Zahlungsbereitschaft für Illustrator geringer ist als der Produktpreis, wird sie es nicht kaufen.

\subsection{2.d}
    %TODO

\subsection{2.e}
    Da die Preise bei einer direkten Kundenbefragung sehr in den Vordergrund rücken und damit verzerrt werden. \\

\subsection{2.f}
    \begin{itemize}
        \item Größe des Bundles: Ausprägung ist die Anzahl Produkte im Bundle
        \item Anzahl der Installationen: Auf wie vielen PCs dürfen die Produkte installiert werden? (1, 2 oder 3)
        \item Preis: Ausprägung sind verschiedene Preisschwellen (49€, 99€, 129€ bspw.)
    \end{itemize}



\section{Aufgabe 3}
\subsection{3.a}
    TV-Werbung erreicht generell sehr viele Leute, gerade bei Sportevents, welche nur im TV übertragen werden. Die Anzahl der erreichten Kontakte durhc Onlinewerbung hängt davon ab, wie oft die Werbung gezeigt wird. \\
    Die Präzision ist bei Onlinewerbung höher als bei TV-Werbung, da durch Nutzerprofile die Kontakte einzeln ausgewählt werden können.

\subsection{3.b}
    \begin{itemize}
        \item Werbebanner: Kleine Banner auf Webseiten mit statischem Inhalt, könnten zB Abbildungen neuer Schuhe zeigen.
        \item Werbevideos: Vor Online-Clips gezeigte kurze Werbevideos könnten kurze Spots von Läufern oder Fussballern zeigen
        \item Werbung bei Suchmaschinen: In den Ergbnissen von Suchmaschinen könnten bei der Suche nach "Sneaker" die Modelle von Adidas ganz oben stehen. 
    \end{itemize}

\subsection{3.c}
    A nutzt die zentrale Route, da nur interessierte Kunden sich den ganzen Text durchlesen werden. Durch gute Argumente wird der Kunde von der Werbung überzeugt. \\
    B nutzt die periphjere Route, da der Kunde nur kurz in den Kontakt mit dem Schuh kommt und nicht viele Informationen gezeigt werden. \\
    \ \\
    Die Printwerbung ist für die zentrale Route besser geeignet, da sie größer gedruckt werden kann. \\
    Bei Onlinewerbung ist meist nur ein kleiner Teil für die Werbung reserviert, das Involvement ist auch sehr viel geringer, da Onlinewerbung als besonders störend empfunden wird.

\subsection{3.d}
    Nein, da wie bereits gesagt Printwerbung nicht für die kleine Fläche geeignet ist. Das Involvement ist zu gering und der Text zu klein. 

\subsection{3.e}
    Durch die Messung von Clicks auf die Werbung kann sehr gut gemessen werden, wie gut die Werbung funktioniert. \\
    Bei der Werbung B könnte beim Click auf "Shop Boost" ein Cookie gesetzt werden, welcher dem Shop verrät wie man auf die Siete gekommen ist.
